\documentclass{article}
\usepackage[utf8]{inputenc}
\usepackage{amsfonts}
\usepackage{mathabx}
\usepackage[shortlabels]{enumitem}
\usepackage[danish]{babel}
\usepackage{amsmath}
\setlength{\parindent}{0em}
\setlength{\parskip}{1.4ex}

\title{Take-home-eksamen i DM500}
\author{Hold 7 Studiegruppe 3.1 \\ Rasmus Burkarl Pedersen, rpede22 \\ Oliver Coppock Pedersen, olped22 \\ Eva Agerbo Rindom, evrin22}
\date{20. November 2022}

\begin{document}

\maketitle

\section{Reeksamen 2015 opgave 2}
\begin{enumerate}[a)]
    \item {Hvilke af følgende udsagn er sande?}
\end{enumerate}

\begin{enumerate} \centering
    \item $ \forall x \in \mathbb{N} : \exists y \in \mathbb{N} : x>y$
    \item $ \forall x \in \mathbb{N} : \exists ! y \in \mathbb{N} : x>y$
    \item $ \exists y \in \mathbb{N} : \forall x \in \mathbb{N} : x>y$
\end{enumerate}
    
\subsection{Delopgave a)}
Nummer 1 er falsk, fordi hvis man sætter $x=0$, så er der ikke noget y der er mindre.
\\
Nummer 2 er falsk, da man godt kan finde flere y'er der er mindre end nogle x'er.
\\
Nummer 3 er falsk, da det skal gælde for alle x'er for et bestemt y og man kan altid finde et x, der er større.

\subsection{Delopgave b)}
\begin{enumerate}[b)]
    \item Angiv negeringen af udsagn 1 fra første spørgsmål
\end{enumerate}
Den så således ud:
\[ \forall x \in \mathbb{N} : \exists y \in \mathbb{N} : x>y\]

Negerings-operatoren må ikke indgå i udsagnet.
Dermed bliver det altså
\[ \exists x \in \mathbb{N} : \forall y \in \mathbb{N} : x \not> y\]

\section{Eksamen 2009 opgave 3}
Opgaven går ud på at betragte mængden S = \verb|{1, 2, 3, 4, ..., 15}|, samt den binære relation på S som er defineret ved:
\begin{equation}
    R = \verb|{|(a, b) | b = 2a\verb|}|
\end{equation}
Samtidig blev det efterspurgt, som en del af takehome-eksamenen, at matrixen for S blev optegnet - dog med S indskrænket til S = \verb|{1, 2, ..., 6}|. Denne er opskrevet herunder i Tabel \ref{tabel:S}

\begin{table}[h!]
\centering
    
\begin{tabular}{|c||c|c|c|c|c|c|}
    \hline
     & 1 & 2 & 3 & 4 & 5 & 6 \\
     \hline
    \hline
    1 & 0& 0& 0& 0& 0& 0 \\
    \hline
    2 & 1& 0& 0& 0& 0& 0\\
    \hline
    3 & 0& 0& 0& 0& 0& 0\\
    \hline
    4 & 0& 1& 0& 0& 0& 0\\
    \hline
    5 & 0& 0& 0& 0& 0& 0\\
    \hline
    6 & 0& 0& 1& 0& 0& 0\\
    \hline
    
\end{tabular}
\caption{opskrivning af indskrænket S}
\label{tabel:S}
\end{table}
\subsection{Delopgave a)}
Denne delopgave går ud på at betragte følgende par 
\begin{center}
    (1, 1),(2, 4), (4, 2), (3, 5), (2, 8)
\end{center}
Ud fra disse par skulle opskrives hvilke par der var en del af relationen R, og samtidig også hvilke der var en del af $R^2$. \\ \\

Den eneste af de opstillede par, hvor element 2 er dobbelt så stor som element 1 er (2, 4), og denne er derved den eneste af parrene, som er en del af R. \\ \\

For at finde ud af hvilke af parrene, som er en del af $R^2$ skal man bare det par, hvor element 2 er dobbelt så stor som det tal, som er dobbelt så stor som element 1 - altså, element 2 er fire gange så stort som element 1. Der er kun et par, som opfylder dette, nemlig parret (2, 8), da det dobbelte af 2 er 4 og det dobbelte af 4 er 8. \\ \\

Da vi herved har fundet det eneste par som er en del af R, nemlig (2, 4) og det eneste par der er en del af $R^2$, nemlig (2, 8), er delopgave a) altså løst.

\subsection{Delopgave b)}
Denne delopgave går ud på at opskrive den transitive lukning af R. \\ \\

Det er værd at huske, transitiv betyder, at hvis relationen indeholder parret (a, b) og (b, c), skal den også indeholde parret (a, c). \\ Samtidig husker vi også, at en transitiv lukning er mængden af par der mangler i en relation, for at denne relation er transitiv. \\ \\

Vi kan derfor først opskrive de par som er indeholdt i R, for derefter at finde dem som skal til, for at R er transitiv. De par, som er indeholdt i R er: \\ \\

\begin{center}
    (1, 2), (2, 4), (3, 6), (4, 8), (5, 10), (6, 12), (7, 14)
\end{center}

Ud fra denne opskrivning kan det observeres, at den transitive lukning må indeholde parrene:

\begin{center}
    (1, 4), (1, 8), (2, 8), (3, 12)
\end{center}

Da vi ved foreningen af R og mængden som indeholder disse par, har en transitiv relation. Opgaven er derfor løst.

\section{Reeksamen 2012 opgave 1}
Betragt funktionerne f:   $\mathbb{R} \rightarrow \mathbb{R}$ og g: $\mathbb{R} \rightarrow \mathbb{R}$ defineret ved
\[f(x)=x^2+x+1 \hspace{1mm}   og\] 
\[g(x)=2x-2\]
\begin{enumerate}[a)]
    \item Er f en bijektion?
    \item Har f en invers funktion?
    \item Angiv $f+g$.
    \item Angiv $g \hspace{1mm} o \hspace{1mm} f$
\end{enumerate}

\subsection{Delopgave a)}

I denne opgave skulle vi finde ud af, om funktionen f er en bijektion. \\
Vi husker at en funktion er bijektiv, hvis og kun hvis den er både injektiv og surjektiv. \\
Ud fra dette husker vi, at en funktion er injektiv så længe f(a) = f(b), så ved vi at a = b. Dette betyder, at to forskellige startværdier ikke kan give den samme funktionsværdi. \\

Vi kan se på funktionens definition f, at denne ikke kan være injektiv, da f(1) og f(-2) giver det samme, bevist herunder:

\begin{equation}
    \begin{aligned}
        f(1) & = 1^2 + 1 + 1 \\
        & = 3 \\ \\
        f(-2) &= (-2)^2 - 2 + 1 \\
        & = 4 - 2 + 1 \\
        & = 3
    \end{aligned}
\end{equation}

Siden f ikke er injektiv, kan denne heller ikke være bijektiv.

\subsection{Delopgave b)}

I denne opgave skulle vi finde ud af, om funktionen f har en invers funktion. \\
Vi husker, at en invers funktion er en funktion, som finder værdien a ud af værdien af f(a) \\
Mere vigtigt, husker vi at en invers funktion kun kan findes, hvis den originale funktion er bijektiv. Siden f ikke er bijektiv, kan vi ikke finde en invers funktion $f^{-1}$
\subsection{Delopgave c)}
Vi skulle angive $f+g$ som regnes på følgende måde:
\[ f+g=x^2+x+1+2x-2\]
Og bliver til:
\[ =x^2+3x-1 \]

\subsection{Delopgave d)}
I delopgave d skulle vi angive $g \hspace{1mm} o \hspace{1mm} f$ som regnes ved at sætte f ind i g:
\[g \hspace{1mm} o \hspace{1mm} f=2(x^2+x+1)-2\]
\[=2x^2+2x+2-2\]
Det bliver derfor til:
\[=2x^2+2x\]

\section{Reeksamen 2015 opgave 1}

Universet er
\[U=\{1,2,3,...,15\}\]

To mængder \(A=\{2n | n \in S\}\) og \(B=\{3n+2 | n \in S\}\)


Hvor \(S=\{1,2,3,4\}\)

Angiv samtlige elementer i hvert af følgende elementer:

a) \(A\) \par

\[A=\{2n | n \in S\}\]
\[A=\{2,4,6,8\}\]
\par
b) \(B\) \par
\[B=\{3n+2 | n \in S\}\]
\[B=\{5,8,11,14\}\]
\par
c) \(A \cap B\) \par
\[A \cap B=\{8\}\]
\par
d) \(A \cup B\) \par
\[A \cup B=\{2,4,5,6,8,11,14\}\]
\par
e) \(A - B\) \par
\[A - B=\{2,4,6\}\]
\par
f) \(\overline A\) \par
\[\overline A=\{1,3,5,7,9,10,11,12,13,14,15\}\]
\par

\end{document}
