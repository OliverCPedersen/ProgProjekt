\documentclass{article}
\usepackage[utf8]{inputenc}
\usepackage{amsfonts}
\usepackage{mathabx}
\usepackage[shortlabels]{enumitem}

\title{Take-home-eksamen i DM500}
\author{Hold 7 Studiegruppe 3.1 \\ Rasmus Burkarl Pedersen, rpede22}
\date{20. November 2022}

\begin{document}

\maketitle

\section{Opgave 1}
\begin{enumerate}[a)]
    \item {Hvilke af følgende udsagn er sande?}
\end{enumerate}

\begin{enumerate} \centering
    \item $ \forall x \in \mathbb{N} : \exists y \in \mathbb{N} : x>y$
    \item $ \forall x \in \mathbb{N} : \exists ! y \in \mathbb{N} : x>y$
    \item $ \exists y \in \mathbb{N} : \forall x \in \mathbb{N} : x>y$
\end{enumerate}
    
\subsection{Svar:}
Nummer 1 er falsk, fordi hvis man sætter $x=0$, så er der ikke noget y der er mindre.
\\
Nummer 2 er falsk, da man godt kan finde flere y'er der er mindre end nogle x'er.
\\
Nummer 3 er falsk, da det skal gælde for alle x'er for et bestemt y og man kan altid finde et x, der er større.

\subsection{Opgave 2}
\begin{enumerate}[b)]
    \item Angiv negeringen af udsagn 1 fra første spørgsmål
\end{enumerate}
Den så således ud:
\[ \forall x \in \mathbb{N} : \exists y \in \mathbb{N} : x>y\]

\subsubsection{Svar - Negerings-operatoren må ikke indgå udsagnet:}
Dermed bliver det altså:
\[ \exists x \in \mathbb{N} : \forall y \in \mathbb{N} : x \not> y\]

\end{document}
